\documentclass[12pt]{article}
\usepackage[T1]{fontenc}
\usepackage[utf8]{inputenc}
\usepackage{polski}

\usepackage{amsmath}
\usepackage{color}
\usepackage{graphicx}

\usepackage{tikz}

\newcommand\encircle[1]{%
  \tikz[baseline=(X.base)] 
    \node (X) [draw, shape=circle, inner sep=1] {\Large{\strut #1}};}

\newcommand\mkcircle[1]{%
  \tikz[baseline=(X.base)] 
    \node (X) [draw, shape=circle, inner sep=0] {\large{\strut #1}};}

\begin{document}
\begin{figure}[h!]
\encircle{$100$}
\put(0,5){\line(1,4){50}}
\put(0,5){\line(2,0){50}}
\put(0,5){\line(1,-4){50}}
%%%%%%%%%%%%%%
\put(55,240){\mkcircle{\textbf{A}}}
\put(50,205){\encircle{$115$}}
\put(85,210){\line(1,1){50}}
\put(85,210){\line(1,0){50}}
\put(85,210){\line(1,-1){50}}
\put(135,265){\encircle{$120$}}
\put(135,210){\encircle{$105$}}
\put(135,160){\encircle{$90$}}
%%%%%%%%%%%%%%
\put(50,0){\encircle{$100$}}
\put(85,5){\line(1,1){50}}
\put(85,5){\line(1,0){50}}
\put(85,5){\line(1,-1){50}}
\put(135,55){\encircle{$110$}}
\put(135,0){\encircle{$85$}}
\put(135,-50){\encircle{$97$}}
%%%%%%%%%%%%%%%%%5
\put(50,-200){\encircle{$90$}}
\put(80,-195){\line(1,1){50}}
\put(80,-195){\line(1,0){50}}
\put(80,-195){\line(1,-1){50}}
\put(130,-145){\encircle{$99$}}
\put(130,-200){\encircle{$95$}}
\put(130,-250){\encircle{$88$}}
%%%%%%%%%%%%%%%%%%
%os czasu
\put(-30,-270){\line(1,0){185}}
\put(-20,-275){\line(0,1){10}}
\put(-22,-285){$t_0$}
\put(65,-275){\line(0,1){10}}
\put(63,-285){$t_1$}
\put(145,-275){\line(0,1){10}}
\put(143,-285){$t_2$}
%%%%%%%%%%%%%%%%%%%5
%podpis
\put(0,-310){(a) przykład drzewa cenowego}
%%%%%%%%%%%%%%%%%%%%%%%%%%%%
\put(205,0){\encircle{$4.60$}}
\put(245,5){\line(1,4){50}}
\put(245,5){\line(2,0){50}}
\put(245,5){\line(1,-4){50}}
%%%%%%%%%%%%%%
\put(295,205){\encircle{$3.25$}}
\put(335,210){\line(1,1){50}}
\put(335,210){\line(1,0){50}}
\put(335,210){\line(1,-1){50}}
\put(385,265){\encircle{$0$}}
\put(385,210){\encircle{$0$}}
\put(385,160){\encircle{$10$}}
%%%%%%%%%%%%%%
\put(295,0){\encircle{$5.83$}}
\put(335,5){\line(1,1){50}}
\put(335,5){\line(1,0){50}}
\put(335,5){\line(1,-1){50}}
\put(385,55){\encircle{$0$}}
\put(385,0){\encircle{$15$}}
\put(385,-50){\encircle{$13$}}
%%%%%%%%%%%%%%%%%5
\put(295,-200){\encircle{$5.28$}}
\put(335,-195){\line(1,1){50}}
\put(335,-195){\line(1,0){50}}
\put(335,-195){\line(1,-1){50}}
\put(385,-145){\encircle{$1$}}
\put(385,-200){\encircle{$5$}}
\put(385,-250){\encircle{$20$}}
%%%%%%%%%%%%%%%%%%%%%%%%%%
%os czasu
\put(210,-270){\line(1,0){200}}
\put(220,-275){\line(0,1){10}}
\put(220,-285){$t_0$}
\put(315,-275){\line(0,1){10}}
\put(313,-285){$t_1$}
\put(400,-275){\line(0,1){10}}
\put(398,-285){$t_2$}
%%%%%%%%%%%
%podpis
\put(220,-310){(b) przykład wyliczenia estymatora $\theta$}
%%%%%%%%%%%%%%%%
\caption{Przykład $\theta$}\label{fig:dolny} 
\end{figure}
\end{document}